%!TEX program = lualatex
%!TEX program = pdflatex
\documentclass[12pt, a4paper]{report}

\usepackage{calc}
\usepackage{eso-pic}
\usepackage{graphicx}
\usepackage{geometry}
\usepackage{titletoc,tocloft}
\usepackage{emoji}
\usepackage{xcolor,mdframed}
\usepackage{blindtext}
\usepackage{tikz}
\usepackage[most]{tcolorbox}
\setlength{\cftchapnumwidth}{1.5cm}
\setcounter{tocdepth}{1}

\geometry{legalpaper, margin=3cm}
\graphicspath{{../images/}}

\AddToShipoutPictureBG{%
	\includegraphics[width=\paperwidth,height=\paperheight]{background.png}
}

\definecolor{red}{RGB}{251,187,187}

\newtcolorbox[auto counter,number within=section]{caja}[1][]{%
  enhanced jigsaw,colback=red,colframe=red,coltitle=red,
  sharp corners,
  leftrule=26mm,
  underlay unbroken and first={\node[below,text=black,anchor=east]
  at ([xshift=2mm]interior.base west) {\includegraphics[scale=0.55]{warnning.png}};},
  %%%%%%%%%%%%%%%%%%%%%%%%
  breakable,pad at break=1mm,
  height=3cm
}

\AtBeginDocument{%
  \addtocontents{toc}{\large}
  \addtocontents{lof}{\large}
}

\newenvironment{important}[1][]{%
	\makebox[0pt][l]
}

\begin{document}
\begin{titlepage}
	\centering
	\title{%
		\vspace{-6cm}
		\includegraphics[scale=1.8]{Logo_42.png} \\
		\vspace{1cm}
		\huge Bangkok Discovery Piscine\\
		\vspace{.6cm}
		\LARGE Welcome Booklet
		}

	% \author{made with \emoji{heart} by tliangso}
	\date{%
		\vspace{2.2cm}
		\it\normalsize Summary: \\
		This Piscine is designed to help students to learn the basics of programming and to get familiar with the 42's learning environment. \\
		The course is divided into 3 parts: Shell, HTML/CSS and JavaScript. \\
		The first part is Shell, which will teach you the basics of using the command line. \\
		The second part is HTML/CSS, which will teach you the basics of web development.\\
		The last part is JavaScript, which will teach you the basics of programming in JavaScript.\\
		\vspace{.8cm}
		Version: 1.0
		}
	\maketitle


	\end{titlepage}
\newpage
	\renewcommand{\thechapter}{\Roman{chapter}}
	{\small \tableofcontents}

\newpage

\chapter{A word about this Discovery Pool}
	\normalsize Welcome !
		\\
	\indent You will begin the first cell of this discovery pool of computer programming. We want
		to both show you what the code is that makes up the software you use every day, and at
		the same time experience peer-learning, an educational model of 42.
		\\
	\indent Programming involves logic (not math). It provides you with elementary bricks, which
		you assemble as you wish. There is never THE solution to a problem. There will be your
		solution, there will be those of each of your neighbors. Slow or fast, ugly or beautiful, if
		that gets the job done, that's all it takes! This assembly of bricks will constitute a series
		of orders (calculation, display, ...) that the computer will perform, in the order you have
		chosen.
		\\
	\indent Rather than giving you a course with only one solution for each problem, and which
		will probably be outdated in a few years, we have chosen to put you in a peer-learning
		situation. You are going to look for the elements that could serve you for your challenge,
		sort out those that are actually interesting by testing and manipulating them, and create
		your own program. To do this, discuss with others, exchange your points of view, find
		new ideas together, and finally test for yourself even to convince you that it works.
		\\
	\indent Peer-evaluation is a key moment to discover other ways of doing things, as well as
		special cases that you have not thought of and that could undermine your program (think
		about your degree of nervousness with software which crashes). Like different clients who
		don't pay attention to the same things, each reviewer will be different from the last. And
		who knows, you might have made new acquaintances for later collaborations.
		\\
	\indent At the end of this pool, you will not have done the same things as the other par-
		ticipants, you will not have validated the same projects, you will have chosen to do one
		challenge rather than another ... .and that's normal! It's both a collective and a personal
		experience. Everyone will benefit from what he or she experiences during this time.
		\\
	\indent Good luck to all, we hope you will like this discovery.

\chapter{General Instructions}
Unless explicitely specified, the following rules will apply every day of this Picine.
	\begin{itemize}
		\item This subject is the one and only trustable source. Don't trust any rumor.
		\item This subject can be updated up to one hour before the turn-in deadline.
		\item The assignments in a subject must be done in the given order. Later assignments
				wont be rated unless all the previous ones are perfectly executed
		\item Be careful about the access rights of your files and folders.
		\item Your assignments will be evaluated by your Piscine peers.
		\item All shell assignments must run on /bin/bash.
		\item You \underline{must not} leave in your turn-in your workspace any file other than the ones
				explicitly requested By the assignments.
		\item You have a question? Ask your left neighbour. Otherwise, try your luck with your
				right neighbor.
		\item Every technical answer you might need is avaiable in the \textbf{man} or on the Internet.
		\item Remember to use the Discord.
		\item You must read the examples thoroughly. They can reveal requirements that are not obvious in
				the assignment's description.
		\item By Thor, by Odin! Use your brain!!!
		\end{itemize}

\chapter{Submission and peer-evaluation}
	\begin{itemize}
		\item \textbf{Submission} \\
			You will have to set your project as complete on \textbf{Intra} and
			upload your work to \textbf{Google Drive}
			folder that will be provided to you.
		\item \textbf{Peer Evaluation} \\
			Both party must be present before evaulation start.
			Evaluation can take place on site only, online evaluation is \textbf{Forbidden}.
			You can and should contact your evaluator or evaulatee before proceeding to cancel the defence.
		\item \textbf{Evaluator Late} \\
			Evaluatee can choose to cancel their evaluation without
			losing evaluation points if the other party are at least 15 min late.
		\item \textbf{Evaluatee Late} \\
			Evaluator begin the evaluation and mark evaluatee as missing.
		\item \textbf{The Evaluation} \\
			Evaluator are encourage to ask evaluatee any question \textbf{related} to the subject.
			Evaluator may ask more question than what the subject ask for.
			If evaluatee cannot defend point(s) or answer question(s) evaluator deem crucial to the subject,
			evaluator can choose to end the evaluation and fail the evaluatee.
		\item \textbf{Post Evaluation} \\
			After the evaluation, evaluator need to record the result in the \textbf{Google Sheet} provided and
			submit the evaluation result on \textbf{Intra} after.
		\end{itemize}

	\begin{caja}
		\vspace{.8cm}
		\fontfamily{lmr}\footnotesize Please note, failure to follow any of the above points will result in instant failure of your project!
		\end{caja}


\end{document}
